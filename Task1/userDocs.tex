\documentclass{article}

\title{Uniting Twist User Documentation}
\author{Nick Werle}

\begin{document}
\maketitle
Uniting Twist is a simple game for android, which combines familiar mechanics to puzzle games (e.g firing balls into a central cluster trying to match them), but provides a novel new way to control those mechanics.

\section{Starting A Game}
When you first launch the game, you will be greeted by a black screen with a white hexagon in the center, and if you have played before, a high score will be written at the bottom of the screen. Simply tap anywhere on the screen to begin the game. The white hexagon will shrink and a gray circle will surround it, marking the beginning of the game.

\section{Objective}
The goal of this game is to make as many matches as possible, without placing any tiles outside of the gray circle on screen. The more matches you make, the higher the score.

\section{Controls}
To play the game, you twist your phone to the left and right. Twisting the phone manipulates the tiles in the center - it rotates the game "board", if you will. Note that the game does \textbf{not} respond to tilting, only twisting, so pick an angle that is comfortable before you begin. The pieces will rotate in from the side - as though they were moving in real space, and you were only manipulating the board itself.

\section{Gameplay}
Pieces will fly into the screen from the side. By twisting your device, you can pair similarly colored tiles together. Pairing 3 or more tiles of the same color increases your score. Furthermore, any disconnected tiles (of ANY color) that are no longer connected are also removed. Doing this, with effective planning, can increase your score very quickly.

As more and more tiles appear, it becomes harder and harder to make matches as you collide with tiles placed previously. Additionally, to turn up the heat, the speed of the tiles flying in increases, which means you much twist much faster to move a piece to the other side.

\section{Losing}
Losing is inevitable. As the game steadily increases the difficulty, you will eventually find yourself creating a stack of tiles which is simply too tall. Once you place a tile beyond the gray circle, you lose lose the game and are presented back at the initial screen with the white hexagon. 

\end{document}
